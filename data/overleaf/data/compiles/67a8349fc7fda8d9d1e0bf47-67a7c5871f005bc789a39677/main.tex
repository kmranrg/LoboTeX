\documentclass{article}
\usepackage{graphicx}

\title{Seminar Report: Turning Word Problems You Can’t Solve into Math Problems You Can: Formulating Robotics Problems as Tractable Constrained Optimizations}
\author{Kumar Anurag}
\date{February 07, 2025}

\begin{document}

\maketitle

\section{Presentation Summary}
I attended a seminar titled \textit{Turning Word Problems You Can’t Solve into Math Problems You Can: Formulating Robotics Problems as Tractable Constrained Optimizations}, presented by Geordan Gutow. The seminar explored the translation of complex engineering problems into mathematical formulations, focusing particularly on constrained optimization techniques used in robotics.

The presentation began by emphasizing the role of engineers in converting real-world problems described in words into mathematical models. These models often take the form of \textbf{constrained optimization problems}, where the goal is to maximize or minimize a particular objective while adhering to design constraints.

Several robotics problems were discussed, demonstrating how they can be reformulated as constrained optimization problems. For instance, the speaker explained how to \textbf{find the distance from a point to the surface of a robotic arm link}, using triangle meshes to represent the link surface. Techniques like \textbf{pruning triangles} and optimizing voxel intersections were introduced to streamline computations.

The seminar covered \textbf{non-contact inspection} techniques for identifying non-visible failures in mechanisms, followed by steps to dock and manipulate the mechanism for diagnostics. Another key topic was the \textbf{integration of functions} that can only be evaluated pointwise, which was reformulated into selecting nodes and weights to create efficient integration schemes. 

Optimization methods like \textbf{Mixed-Integer Linear Programming (MILP)} and \textbf{Second-Order Cone Programming (SOCP)} were highlighted for their practical efficiency despite theoretical complexity. The seminar demonstrated how \textbf{convex relaxations} can transform otherwise intractable problems into solvable forms.

Applications in \textbf{satellite docking} were explored, focusing on planning safe, low-thrust approach trajectories. The seminar detailed the transformation of non-convex docking problems into convex forms, allowing for reliable optimization.

The presentation also covered \textbf{multi-agent pathfinding} and \textbf{abstract action planning}, where problems like robotic assembly of modular structures were formulated as graph problems. Concepts like \textbf{Action Dependency Graphs (ADG)} and \textbf{Conflict-Based Search (CBS)} were employed to coordinate multi-robot systems efficiently.

The seminar concluded with discussions on \textbf{0-g assembly} and \textbf{Moving Target Traveling Salesman Problems (TSP)}, showcasing methods for optimizing interception sequences and paths using techniques like \textbf{Hybridization over Parallel Generalized TSPs (HPG)}.

\section{What I Learned}
Attending this seminar provided me with a deeper understanding of how \textbf{constrained optimization} techniques can be applied to solve complex robotics problems. I learned that many seemingly distinct problems can be reformulated into a common mathematical framework, making them more tractable.

One of the key insights I gained was the practical application of \textbf{Mixed-Integer Linear Programming (MILP)} and \textbf{Second-Order Cone Programming (SOCP)}. Despite their theoretical challenges, these methods can be efficiently solved in practice using techniques like \textbf{branch-and-cut} and \textbf{sequential convex programming}. This understanding will be valuable for addressing real-world engineering problems in my future research.

I also learned about the importance of \textbf{convex relaxation} in simplifying non-convex problems, such as trajectory planning for satellite docking. By transforming these problems into convex forms, it becomes possible to find reliable solutions while ensuring safety constraints are met.

The seminar introduced me to novel concepts in \textbf{multi-agent systems} and \textbf{abstract action planning}. I now appreciate how \textbf{Action Dependency Graphs (ADG)} and \textbf{Conflict-Based Search (CBS)} can be used to coordinate complex tasks involving multiple robots, such as the assembly of modular structures.

Furthermore, the discussion on \textbf{Moving Target Traveling Salesman Problems (TSP)} and techniques like \textbf{Hybridization over Parallel Generalized TSPs (HPG)} broadened my perspective on optimization in dynamic environments. These methods can significantly enhance the efficiency of robotic operations, especially in scenarios involving moving targets and obstacles.

Overall, this seminar enriched my knowledge of formulating and solving robotics problems using advanced optimization techniques. The insights I gained will undoubtedly influence my approach to research and problem-solving in robotics and engineering.

\end{document}
